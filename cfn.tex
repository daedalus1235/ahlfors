%! TEX root = mmyau.tex
\chapter{Complex Functions}\label{ch:cfn}
\section{Introduction to the Concept of Analytic Function}
\subsection{Limits and Continuity}

\subsection{Analytic Functions}
\subsubsection{}
\subsubsection{}
\subsubsection{}
\subsubsection{do this}
Suppose \(f(z)\) is analytic and \(\abs{f(z)}=k\,\forall z\in\C\).
\subsubsection{do this}
WLOG, suppose \(f(z)\) is analytic. 
\subsubsection{}
\subsubsection{optional}
Suppose \(u(x,y)\) is harmonic; that is \(\pd_x^2 u + \pd_y^2 u = 0\). We write the ``change of coordinates''
\[2x = z+\bar z\qquad 2iy = z-\bar z\]
so
\[\pder{u}{z} = \pder{u}{x}\pder{x}{z}+\pder{u}{y}\pder{y}{z} = \frac{1}{2}\pder{u}{x}-\frac{i}{2}\pder{u}{y}\]
and similarly,
\[\pder{u}{\bar z} = \pder{u}{x}\pder{x}{\bar z}+\pder{u}{y}\pder{y}{\bar z} = \frac{1}{2}\pder{u}{x}+\frac{i}{2}\pder{u}{y}\]
we then have
\begin{align*}
	\pder{}{z}\pder{u}{\bar z} &= \pder{}{z}\frac{1}{2}\pder{u}{x}+\pder{}{z}\frac{i}{2}\pder{u}{y}\\
				   &=\frac{1}{4}\pder{^2u}{x^2}-\frac{i}{2}\pder{^2u}{y\,\pd x} + \frac{i}{2}\pder{^2u}{x\,\pd y}+\frac{1}{4}\pder{^2u}{y^2}\\
				   &=0
\end{align*}
because mixed partials are probably symmetric for harmonic functions.
\subsection{Polynomials}

\subsection{Rational Functions}
\subsubsection{}
\subsubsection{}
\subsubsection{}
\subsubsection{}
\subsubsection{}
\subsubsection{}

\section{Elementary Theory of Power Series}
\subsection{Sequences}

\subsection{Series}

\subsection{Uniform Convergence}
\subsubsection{do this}
\subsubsection{optional}
\subsubsection{}
\subsubsection{}
\subsubsection{}
\subsubsection{do this}

\subsection{Power Series}
\subsubsection{}
\subsubsection{}
\subsubsection{do this}
\subsubsection{do this}
\subsubsection{optional}
\subsubsection{}
\subsubsection{}
\subsubsection{}
\subsubsection{}

\subsection{Abel's Limit Theorem}

\section{The Exponential and Trigonometric Functions}
\subsection{The Exponential}

\subsection{The Trigonometric Functions}
\subsubsection{}
\subsubsection{}
\subsubsection{}
\subsubsection{}

\subsection{The Periodicity}

\subsection{The Logarithm}
\subsubsection{}
\subsubsection{optional}
\subsubsection{}
\subsubsection{}
\subsubsection{}
\subsubsection{}
\subsubsection{}
\subsubsection{do this}
\subsubsection{}
\subsubsection{}
