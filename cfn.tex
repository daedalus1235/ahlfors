%! TEX root = mmyau.tex
\chapter{Complex Functions}\label{ch:cfn}
\section{Introduction to the Concept of Analytic Function}
\subsection{Limits and Continuity}

\subsection{Analytic Functions}
\subsubsection{}
\subsubsection{}
\subsubsection{}
\subsubsection{do this}
Suppose \(f(z)\) is analytic and WLOG that \(\abs{f(z)}=1\,\forall z\in\C\).
Write
\[f(z)=u(x,y)+v(x,y)\qquad\qquad z=x+iy\]
Then, we have 
\[u^2+v^2=1\]
\[\del^2u=\del^2v=0\]
and the Cauchy-Riemann equations
\[\pder{u}{x} =\pder{v}{y}\qquad\pder{v}{x}=-\pder{u}{y}\]
Taking the derivatives of \(\abs{f(z)}^2=1\), we find
\[\left(\pder{u}{x}\right)^2+u\pder{^2u}{x^2}+v\pder{^2v}{x^2}+\left(\pder{v}{x}\right)^2=0\]
\[\left(\pder{u}{y}\right)^2+u\pder{^2u}{y^2}+v\pder{^2v}{y^2}+\left(\pder{v}{y}\right)^2=0\]
Adding together and using the harmonicity of \(u\) and \(v\) to simplify,
\[\left(\pder{u}{x}\right)^2+\left(\pder{u}{y}\right)^2+\left(\pder{v}{x}\right)^2+\left(\pder{v}{y}\right)^2=0\]
Usign the Cauchy-Riemann equations, we have
\[\left(\pder{u}{x}\right)^2 = \left(\pder{v}{y}\right)\qquad\qquad\left(\pder{u}{y}\right)^2=\left(\pder{v}{x}\right)^2\]
so we can simplify to find that
\[\left(\pder{u}{x}\right)^2+\left(\pder{u}{y}\right)^2=0\]
so
\[\abs{f'(z)}^2=0\then f'(z)=0\]
or \(f(z)\) must be a constant.
\subsubsection{do this}
Suppose \(f(z)\) is analytic. It has a derivative characterized by the existence of 
\[\pder{u}{x},\pder{u}{y},\pder{v}{x},\pder{v}{y}\]
which satisfy the Cauchy-Riemann equations
\[\pder{u}{x}=\pder{v}{y}\qquad \pder{v}{x} = -\pder{u}{y}\]
we can thus write 
\[\der{f}{z} = \pder{u}{x}+i\pder{v}{x}\]
Let 
\[\bar f (\bar z) = u-iv = u+iv'(x,y')\]
where \(v'=-v\) and \(y' = -y\).
To show that \(\bar f(\bar z)\) is analytic, we need only show that \(u\) and \(v'(x,y')\) are harmonic conjugates. First, we show that \(v'\) is harmonic:
\[\del^2v'=\pder{^2v'}{x^2}+\pder{}{y}\pder{v'}{y} = \pder{^2v}{x^2}+\pder{}{y}\pder{v}{y}=\del^2v=0\]
now we must show that \(v'\) is conjugate to \(u\); that is it satisfies the Cauchy-Riemann equations. We saw that we can write
\[\pder{v'}{x}=\pder{v}{x} =-\pder{u}{y}\]
\[\pder{v'}{y}=\pder{v}{y} =\pder{u}{x}\]
thus \(f(z)\) is analytic iff \(\bar f(\bar z)\) is analytic.
\subsubsection{}
\subsubsection{optional}
Suppose \(u(x,y)\) is harmonic; that is \(\pd_x^2 u + \pd_y^2 u = 0\). We write the ``change of coordinates''
\[2x = z+\bar z\qquad 2iy = z-\bar z\]
so
\[\pder{u}{z} = \pder{u}{x}\pder{x}{z}+\pder{u}{y}\pder{y}{z} = \frac{1}{2}\pder{u}{x}-\frac{i}{2}\pder{u}{y}\]
and similarly,
\[\pder{u}{\bar z} = \pder{u}{x}\pder{x}{\bar z}+\pder{u}{y}\pder{y}{\bar z} = \frac{1}{2}\pder{u}{x}+\frac{i}{2}\pder{u}{y}\]
we then have
\begin{align*}
	\pder{}{z}\pder{u}{\bar z} &= \pder{}{z}\frac{1}{2}\pder{u}{x}+\pder{}{z}\frac{i}{2}\pder{u}{y}\\
				   &=\frac{1}{4}\pder{^2u}{x^2}-\frac{i}{2}\pder{^2u}{y\,\pd x} + \frac{i}{2}\pder{^2u}{x\,\pd y}+\frac{1}{4}\pder{^2u}{y^2}\\
				   &=0
\end{align*}
because mixed partials are probably symmetric for harmonic functions.
\subsection{Polynomials}

\subsection{Rational Functions}
\subsubsection{}
\subsubsection{}
\subsubsection{}
\subsubsection{}
\subsubsection{}
\subsubsection{}

\section{Elementary Theory of Power Series}
\subsection{Sequences}

\subsection{Series}

\subsection{Uniform Convergence}
\subsubsection{do this}
A sequence \(\{a_n\}\) converges to \(a\) iff \(\forall\varepsilon>0\,\exists N\) s.t. \(\abs{a-a_n}<\varepsilon\,\forall n\geq N\). Fix \(N\) such that it satisfies \(\varepsilon=1\). The sequence \(\{a_n\}\) is then bounded by
\[r=\max(1,\abs{a-a_1},\ldots,\abs{a-a_N})\]
\subsubsection{optional}
\subsubsection{}
\subsubsection{}
\subsubsection{}
\subsubsection{do this}
Let 
\[U_n = \sum_{i=1}^n u_i\qquad\qquad V_n = \sum_{i=1}^n v_i\]
and WLOG let \(U\) converge absolutely. 
Let 
\[P_n=U_nV_n=\sum_{i=1}^np_i\]
where
\[p_n = \sum_{i=1}^{n-1} u_i v_{n-i}\]
we thus want to show that \(P_n\) converges to some \(P\).

Let \(\delta_n = V_n-V\). Then, we can rewrite
\begin{align*}
	P_n &= u_1v_1 + (u_1v_2 + u_2v_1) + (u_1v_3+u_2v_2+u_3v_1)+\cdots\\
	&= u_1 V_n + u_2 V_{n-1} + \cdots+ u_n V_1\\
	&=u_1(V+\delta_n)+\cdots + u_n (V+\delta_1)\\
	&= U_nV + u_1\delta_n + u_2\delta_{n-1}+\cdots u_n\delta_1
\end{align*}
Let \(R = u_1\delta_n + \cdots + u_n \delta_1\). Because \(V\) converges, \(\delta_n\to 0\). Choose \(N\) such that \(\delta_{n}<\epsilon\) for \(n\geq N\). Then, 
\begin{align*}
	\abs{R} = \abs{u_n\delta_1+\cdots+u_N\delta_{n+1-N}+u_{N-1}\delta_{n-N}+\cdots+}
\end{align*}


\subsection{Power Series}
\subsubsection{}
\subsubsection{}
\subsubsection{do this}
\subsubsection{do this}
\subsubsection{optional}
\subsubsection{}
\subsubsection{}
\subsubsection{}
\subsubsection{}

\subsection{Abel's Limit Theorem}

\section{The Exponential and Trigonometric Functions}
\subsection{The Exponential}

\subsection{The Trigonometric Functions}
\subsubsection{}
\subsubsection{}
\subsubsection{}
\subsubsection{}

\subsection{The Periodicity}

\subsection{The Logarithm}
\subsubsection{}
\subsubsection{optional}
\subsubsection{}
\subsubsection{}
\subsubsection{}
\subsubsection{}
\subsubsection{}
\subsubsection{do this}
\subsubsection{}
\subsubsection{}
