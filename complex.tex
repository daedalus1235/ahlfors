%! TEX root = mmyau.tex
\chapter{Complex Numbers}\label{ch:complex}
\section{The Algebra of Complex Numbers}
\subsection{Arithmetic Operations}
\subsubsection{}
\[(1+2i)^3 = 1+6i - 12-8i = \boxed{-11-2i}\]
\[\frac{5}{-3+4i} = \frac{-15-20i}{25} = \boxed{-\frac{3}{5}-\frac{4}{5}i}\]
\[\left(\frac{2+i}{3-2i}\right)^2 = \left(\frac{4+7i}{13}\right)^2=\boxed{-\frac{33}{169}+ \frac{56}{169}i}\]
From the binomial expansion of the LHS, and cancelling odd powers of \(i\),
\[(1+i)^n+(1-i)^n=2\sum_{m=0}^{n/2} \binom{n/2}{2m}(-1)^m\]
\subsubsection{}
\[\Re{z^4} = x^4-6x^2y^2+y^4\]
\[\Re\frac1z=\frac{x}{x^2+y^2}\]
\[\Re\frac{z-1}{z+1} = \frac{x^2-1}{(x+1)^2+y^2}\]
\[\Re\frac{1}{z^2} = \Re\frac{1}{x^2-y^2+2xyi}=\frac{x^2-y^2}{(x^2+y^2)^4}\]
\subsubsection{}
\[\left(\frac{-1\pm i \sqrt{3}}{2}\right)^3 = -\frac{1}{8}\pm \frac{3\sqrt{3}}{8}i+\frac{9}{8}\mp\frac{3\sqrt{3}}{8}i=1\]
\[\left(\frac{\pm1\pm i\sqrt{3}}{2}\right)^6=\frac{1}{64}+\frac{6\sqrt{3}}{64}i-\frac{45}{64}-\frac{60\sqrt{3}}{64}i+\frac{135}{64}+\frac{54\sqrt{3}}{64}i-\frac{27}{64}\]
\[\left(\frac{\pm1\mp i\sqrt{3}}{2}\right)^6=\frac{1}{64}-\frac{6\sqrt{3}}{64}i-\frac{45}{64}+\frac{60\sqrt{3}}{64}i+\frac{135}{64}-\frac{54\sqrt{3}}{64}i-\frac{27}{64}\]
\subsection{Square Roots}
\subsubsection{}
\begin{enumerate}[label = (\alph*)]
	\item
		\[a^2-b^2=0\quad 2ab=1\then a=b=\pm\frac{1}{\sqrt{2}}\then \sqrt{i} = \pm\left(\frac{1}{\sqrt{2}}+\frac{i}{\sqrt{2}}\right)\]
	\item
		\[a^2-b^2=0\quad 2ab=-1\then a=b=\pm\frac{i}{\sqrt{2}}\then \sqrt{-i} = \pm \left(-\frac{1}{\sqrt{2}}+\frac{i}{\sqrt{2}}\right)\]
		\item 
			sub \(b=1/2a\) so
\[a^2 = \frac{1}{2}\pm\frac{1}{\sqrt{2}}\]
\[b^2 = -\frac{1}{2}\pm\frac{1}{2}\sqrt{2}\]
enforcing the condition that
\[ab = \frac{1}{2}\]
we obtain
\[\pm \left(\sqrt{\frac{1}{2}+\frac{1}{\sqrt{2}}}+i\sqrt{-\frac{1}{2}+\frac{1}{\sqrt{2}}}\right)\]
\item I really cannot be bothered to do this\ldots
	\[a^2-b^2 = \frac{1}{2}\qquad ab = \frac{\sqrt{3}}{4}\]
	\[(a^2-b^2)^2 = \frac{1}{4}\qquad (a^2+b^2)^2 = 1\]
	\[a^2 = \frac{3}{4}\qquad\qquad b^2 = \frac{1}{4}\]
	Thus,
	\[\sqrt{\frac{1-i\sqrt{3}}{2}} = \pm\left(\frac{\sqrt{3}}{2}-\sqrt{i}{2}\right)\]
\end{enumerate}
\subsubsection{}
Cuz i'm lazy:
\[\frac{1}{\sqrt{2}}+\frac{i}{\sqrt{2}},-\frac{1}{\sqrt{2}}+\frac{i}{\sqrt{2}},-\frac{1}{\sqrt{2}}-\frac{i}{\sqrt{2}},\frac{1}{\sqrt{2}}-\frac{i}{\sqrt{2}}\]
\subsubsection{}
do i really have to

using the fact that \(\sqrt{i} = \pm\left(\frac{1}{\sqrt{2}}+\frac{i}{\sqrt{2}}\right)\), \(\sqrt{i} = \mp\frac{1}{\sqrt{2}}\pm\frac{i}{\sqrt{2}}\), and \(i^4=1\),
\[\sqrt{\sqrt{\pm i}}=a_\pm+ib_\pm\then a_\pm^2-b_\pm^2 = \frac{1}{\sqrt{2}}\qquad a_\pm b_\pm=\frac{1}{2\sqrt{2}}\]
\[a^4_\pm - 2a_\pm^2b_\pm^2 + b_\pm^4 = \frac{1}{2}\then (a_\pm^2+b_\pm^2)^2 = 1\then a_\pm^2+b_\pm^2=1\]
\[\then a_\pm^2 = \frac{1}{2}+\frac{1}{2\sqrt{2}}\qquad b_\pm^2 = \frac{1}{2}-\frac{1}{2\sqrt{2}}\]
\begin{align*}
	\sqrt{\sqrt{i}}&= \sqrt{\frac{1}{2}+\frac{1}{2\sqrt{2}}}+i\sqrt{\frac{1}{2}-\frac{1}{2\sqrt{2}}},\, -\sqrt{\frac{1}{2}-\frac{1}{2\sqrt{2}}}+i\sqrt{\frac{1}{2}+\frac{1}{2\sqrt{2}}},\\
		       &\quad-\sqrt{\frac{1}{2}+\frac{1}{2\sqrt{2}}}-i\sqrt{\frac{1}{2}-\frac{1}{2\sqrt{2}}},\,\sqrt{\frac{1}{2}-\frac{1}{2\sqrt{2}}}-i\sqrt{\frac{1}{2}+\frac{1}{2\sqrt{2}}}\\
	\sqrt{\sqrt{-i}}&= -\sqrt{\frac{1}{2}+\frac{1}{2\sqrt{2}}}+i\sqrt{\frac{1}{2}-\frac{1}{2\sqrt{2}}},\, -\sqrt{\frac{1}{2}-\frac{1}{2\sqrt{2}}}-i\sqrt{\frac{1}{2}+\frac{1}{2\sqrt{2}}},\\
		       &\quad\sqrt{\frac{1}{2}+\frac{1}{2\sqrt{2}}}-i\sqrt{\frac{1}{2}-\frac{1}{2\sqrt{2}}},\,\sqrt{\frac{1}{2}-\frac{1}{2\sqrt{2}}}+i\sqrt{\frac{1}{2}+\frac{1}{2\sqrt{2}}}\\
\end{align*}
\subsubsection{}
are you serious

Plugging into the quadratic formula,
\[z=\frac{-\alpha-i\beta\pm\sqrt{\alpha^2-\beta^2+i2\alpha\beta - 4\gamma - i4\delta}}{2}\]
Taking the square root,
\[a^2-b^2=\alpha^2-\beta^2-4\gamma\]
\[ab = \alpha\beta - 2\delta\]
\[(a^2-b^2)^2 = \alpha^4+\beta^4+16\gamma^2-2\alpha^2\beta^2-8\alpha^2\gamma+8\beta^2\gamma\]
\[a^2b^2 = \alpha^2\beta^2+4\delta^2-4\alpha\beta\delta\]
\[(a^2+b^2)^2 = \alpha^4+\beta^4+16\gamma^2 + 2\alpha^2\beta^2-8\alpha^2\gamma + 8\beta^2\gamma + 16\delta^2 - 16\alpha\beta\delta\]

\[a=\frac{\sqrt{\alpha^2-\beta^2-4\gamma + \sqrt{(\alpha^2+\beta^2)^2+8\gamma(2\gamma - a^2+\beta^2) + 16\delta(\delta-\alpha\beta)}}}{2}\]
\[b=\frac{\sqrt{-\alpha^2+\beta^2+4\gamma + \sqrt{(\alpha^2+\beta^2)^2+8\gamma(2\gamma - a^2+\beta^2) + 16\delta(\delta-\alpha\beta)}}}{2}\]
for
\[z = \frac{-\alpha\pm 2a}{2} -i \frac{\beta\pm 2b}{2}\]
where i literally cannot be bothered to try and fit the above in one single expression.

\subsection{Justification}
\subsubsection{}
Let capital members denote matrices and lower case members denote complex numbers. For a relation \(f:Z\mapsto z\) to be a homomomorphism it must obey \(f(E_+) = e_+\) and \(f(E_\times) = e_\times\), that is, we must have
\[f\left[\begin{pmatrix}0&0\\0&0\end{pmatrix}\right] = 0\quad f\left[\begin{pmatrix}1&0\\0&1\end{pmatrix}\right] = 1\]
Further, we use the fact that
\[\begin{pmatrix}
	0&1\\-1&0
\end{pmatrix} \begin{pmatrix}
	0&1\\-1&0
\end{pmatrix} = \begin{pmatrix}
	1&0\\0&1
\end{pmatrix}\]
to fix 
\[f\left[ \begin{pmatrix}
		0&1\\-1&0
\end{pmatrix}\right]=i\]
arbitrarily. We can thus identify 
\[f\left[ \begin{pmatrix}
		\alpha &\beta\\-\beta&\alpha
\end{pmatrix}\right] = \alpha+i\beta\]
with the inverse map
\[f^{-1}[\alpha+i\beta] = \begin{pmatrix}
	\alpha & \beta \\ -\beta & \alpha
\end{pmatrix}\]
We show that this is a field homomorphism. Let \(f(Z) = \alpha+i\beta\) and \(f(W) = \gamma+i\delta\). Then,
\begin{align*}
	f(Z+W) &= f\left[ \begin{pmatrix}
		\alpha & \beta\\
		-\beta & \alpha
	\end{pmatrix}+ \begin{pmatrix}
		\gamma & \delta\\
		-\delta & \gamma
	\end{pmatrix}\right] \\
	&= f\left[ \begin{pmatrix}
		\alpha+\gamma & \beta + \delta \\
		-(\beta + \delta) & \alpha + \delta
	\end{pmatrix}\right]\\
	&= (\alpha+\gamma)+i(\beta+\delta) \\
	&= (\alpha+i\beta)+(\gamma+i\delta) \\
	&= f(Z)+f(W)
\end{align*}
thus addition is respected. Similarly,
\begin{align*}
	f(ZW) &= f\left[ \begin{pmatrix}
		\alpha & \beta\\
		-\beta & \alpha
	\end{pmatrix} \begin{pmatrix}
		\gamma & \delta\\
		-\delta & \gamma
	\end{pmatrix}\right] \\
	&= f\left[ \begin{pmatrix}
		\alpha\gamma-\beta\delta & \alpha\delta + \beta\gamma\\
		-(\alpha\gamma+\beta\delta) & \alpha\gamma + \beta\gamma
	\end{pmatrix}\right]\\
	&=(\alpha\gamma-\beta\delta)+i(\alpha\delta+\beta\gamma)\\
	&=(\alpha + i\beta)(\gamma+i\delta)\\
	&= f(Z)f(W)
\end{align*}
thus multiplication is also respected. Thus, \(f\) is a field homomorphism.

We can further see that \(f\) is a bijection; it is a surjection because all \(z\in\C\) can be written \(z=\alpha+i\beta\) for \(\alpha,\beta\in \R\), and it clearly an injection because \(f(Z)=f(W)\implies f(Z)-f(W)=0\implies f(Z-W)=0\implies Z-W=0\implies Z=W\). 

Thus, as \(f\) is a field homomorphism and a bijection, it is an isomorphism, and these matrices equiped with matrix addition and matrix multiplication is isomorphic to the complex field. 
\subsubsection{}
I have no idea what this means
\subsection{Conjugation, Absolute Value}
\subsubsection{}
\begin{align*}
	\frac{z}{z^2+1}&= \frac{x+iy}{x^2-y^2+i2xy+1}\\
		       &=\frac{x^3-xy^2-2xy^2+x+i(x^2y-y^3-2x^2y+y)}{x^4-6x^2y^2 + y^4+x^2-y^2+1}\\
	\frac{\bar z}{\bar z^2+1}&= \frac{x-iy}{x^2-y^2-i2xy+1}\\
		       &=\frac{x^3-xy^2-2xy^2+x-i(x^2y-y^3-2x^2y+y)}{x^4-6x^2y^2 + y^4+x^2-y^2+1}\\
		       &=\overline{\left(\frac{z}{z^2+1}\right)}
\end{align*}
\subsubsection{}
Splitting into terms,
\begin{enumerate}[label = (\alph*)]
	\item \(2*\sqrt{10}*\sqrt{20}\sqrt{2}=40\)
	\item \(5*\sqrt{5}/\sqrt{2}\sqrt{10} = \frac{5}{2}\)
\end{enumerate}
\subsubsection{}
\begin{align*}
	\abs{\frac{a-b}{1-\bar a b}}&=\frac{a-b}{1-\bar a b}\frac{\bar a - \bar b}{1-a \bar b}\\
				    &=\frac{\bar a a + \bar b b -\bar a b - a\bar b}{1-\bar a b -a\bar b + \bar a a\bar b b}\\
				    &=\frac{\abs{a}^2+\abs{b}^2-2\Re{\bar a b}}{1+\abs{a}^2\abs{b}^2 - 2\Re{\bar a b}}
\end{align*}
We see that if either \(\abs a = 1\) or \(\abs b = 1\) that the numerator equals the denominator and the fraction cancels. In the case where \(\abs a = \abs b = 1\), the expression still holds so long as \(\Re \bar a b\neq 1\), that is \(a\neq b\).
\subsubsection{}
Make the substitution
\[\alpha = a+b\qquad\qquad \beta = a-b\qquad\then\qquad a = \frac{\alpha+\beta}{2}\qquad\qquad b = \frac{\alpha-\beta}{2}\]
so
\[\alpha\Re z + i\beta\Im z = -c\]
\[\bar \alpha \Re z - i \bar \beta \Im z = -\bar c\]
Adding,
\[\Re\alpha \Re z -\Im \beta \Im z = -\Re c\]
and subtracting,
\[\Im \alpha\Re z +\Re\beta\Im z = -\Im c\]

Cases:
if \(\alpha\) is real, 
\[\Im z = -\frac{\Im c}{\Re \beta}\]
if \(\alpha\) is imaginary,
\[\Im z = -\frac{\Re c}{\Im \beta}\]
if \(\beta\) is real, 
\[\Re z = -\frac{\Re c}{\Re \alpha}\]
if \(\beta\) is imaginary,
\[\Re z = -\frac{\Im c}{\Im\alpha}\]
Note that there is no solution if one of \(\alpha,\beta\) is real and the other is purly imaginary. Further, if either \(\alpha\) or \(\beta\) is zero, we have either infinitely many solutions, characterized by a line, or no solutions.

Finally, consider the case where \(\alpha,\beta\) are nonzero and have both imaginary and complex components. Solving,
\[\Re z = -\frac{\Re \beta\Re c+\Im\beta\Im c}{\Re\alpha\Re\beta +\Im\alpha\Im\beta}\]
\[\Im z = -\frac{\Re\alpha\Im c -\Im\alpha\Re c}{\Re\alpha\Re\beta + \Im\alpha\Im\beta}\]
We see there is a unique solution so long as \(\Re\alpha\Re\beta+\Im\alpha\Im\beta\neq 0\), or if \(\alpha,\beta\) are both real or both imaginary.

\subsubsection{}
Trivially, Lagrange's identity holds for \(n=1\):
\[\abs{a_1b_1}=\abs{a_1}\abs{b_1}+0\]
First, note that through multiplying conjugates, we obtain
\[\sum_{i=1}^{n+1}\abs{a_i\bar b_{n+1}-a_{n+1}\bar b_i}^2=\abs{b_{n+1}}^2\sum_{i=1}^{n+1}\abs{a_i}^2+\abs{b_{n+1}}^2\sum_{i=1}^{n+1}\abs{a_i}^2-2\Re\left[\bar a_{n+1}\bar b_{n+1}\sum_{i=1}^{n+1}a_ib_i\right]\]
Thus, if Lagrange's identity holds for some \(n\),
\begin{align*}
	\abs{\sum_{i=1}^{n+1}a_ib_i}^2&=\abs{\sum_{i=1}^na_ib_i+a_{n+1}b_{n+1}}^2\\
				      &=\abs{\sum_{i=1}^{n}a_i b_i}^2 + \abs{a_{n+1}}^2\abs{b_{n+1}}^2+2\Re\left[\bar a_{n+1}\bar b_{n+1}\sum_{i=1}^{n}a_ib_i\right]\\
				      &=\sum_{i=1}^{n}\abs{a_i}^2\sum_{i=1}^n\abs{b_i}^2 - \sum_{1\leq i < j \leq n}\abs{a_i\bar b_j - a_j\bar b_i}^2+ \abs{a_{n+1}}^2\abs{b_{n+1}}^2\\
				      &\qquad+2\Re\left[\bar a_{n+1}\bar b_{n+1}\sum_{i=1}^{n}a_ib_i\right] + \abs{a_{n+1}}^2\sum_{i=1}^n\abs{b_i} +\abs{b_{n+1}}^2\sum_{i=1}^{n+1}\abs{a_i}^2 \\
				      &\qquad - \abs{a_{n+1}}^2\sum_{i=1}^n\abs{b_i} -\abs{b_{n+1}}^2\sum_{i=1}^{n+1}\abs{a_i}^2- \abs{a_{n+1}}^2\abs{b_{n+1}}^2+ \abs{a_{n+1}}^2\abs{b_{n+1}}^2\\
				      &=\sum_{i=1}^{n}\abs{a_i}^2\sum_{i=1}^n\abs{b_i}^2 + \abs{a_{n+1}}^2\sum_{i=1}^n\abs{b_i}^2 +\abs{b_{n+1}}^2\sum_{i=1}^{n+1}\abs{a_i}^2 \\
				      &\qquad - \sum_{1\leq i < j \leq n}\abs{a_i\bar b_j - a_j\bar b_i}^2\\
				      &\qquad -\abs{a_{n+1}}^2\abs{b_{b+1}}^2- \abs{a_{n+1}}^2\sum_{i=1}^n\abs{b_i}^2 -\abs{b_{n+1}}^2\sum_{i=1}^{n+1}\abs{a_i}^2\\
				      &\qquad+2\Re\left[\bar a_{n+1}\bar b_{n+1}\sum_{i=1}^{n}a_ib_i\right] +\underbrace{2\abs{a_{n+1}}^2\abs{b_{n+1}}^2}_{=2\Re\left[\bar a_{n+1}\bar b_{n+1}a_{n+1}b_{n+1} \right]}\\
				      &=\sum_{i=1}^{n+1}\abs{a_i}^2\sum_{i=1}^{n+1}\abs{b_i}^2 - \sum_{1\leq i < j \leq n}\abs{a_i\bar b_j - a_j\bar b_i}^2\\
				      &\qquad -\abs{a_{n+1}}^2\sum_{i=1}^{n+1}\abs{b_i}^2-\abs{b_{n+1}}^2\sum_{i=1}^{n+1}\abs{b_i}^2-2\Re\left[\bar a_{n+1}\bar b_{n+1}\sum_{i=1}^{n+1}a_ib_i\right]\\
				      &=\sum_{i=1}^{n+1}\abs{a_i}^2\sum_{i=1}^{n+1}\abs{b_i}^2 - \sum_{1\leq i < j \leq n}\abs{a_i\bar b_j - a_j\bar b_i}^2 -\sum_{i=1}^{n+1}\abs{a_i\bar b_{n+1}-a_{n+1}\bar b_i}^2\\
				      &=\sum_{i=1}^{n+1}\abs{a_i}^2\sum_{i=1}^{n+1}\abs{b_i}^2 - \sum_{1\leq i < j \leq n+1}\abs{a_i\bar b_j - a_j\bar b_i}^2
\end{align*}
it also holds for \(n+1\). Thus, Lagrange's identity holds for all \(n\in \N\).
\subsection{Inequalities}
\subsubsection{}
From 1.1.4.3, we can write
\begin{align*}
	\abs{\frac{a-b}{1-\bar a b}}&=\frac{\abs{a}^2 + \abs b^2 - 2\Re\bar a b}{1+\abs a ^2 \abs b ^2 - 2\Re\bar a b}\\
				    &=
\end{align*}
\subsubsection{}
Cauchy's inequality holds as an equality for \(n=1\):
\[\abs{a_1b_1}^2=\abs{a_1}^2\abs{b_1}^2\]
Suppose Cauchy's inequality holds for \(n\). Then,
\begin{align*}
	\abs{\sum_{i=1}^{n+1} a_ib_i}^2&\leq \abs{\sum_{i=1}^n a_ib_i}^2+\abs{a_{n+1}}^2\abs{b_{n+1}}^2\\
				       &\leq \sum_{i=1}^{n}\abs{a_i}^2\sum_{i=1}^{n}\abs{b_i}^2+\abs{a_{n+1}}^2\abs{b_{n+1}}^2\\
				       &\leq \sum_{i=1}^{n}\abs{a_i}^2\sum_{i=1}^{n}\abs{b_i}^2+\abs{a_{n+1}}^2\abs{b_{n+1}}^2+\abs{a_{n+1}}\sum_{i=1}^{n}\abs{b_i}^n + \abs{b_{n+1}}^2\sum_{i=1}^n\abs{a_i}^2\\
				       &=\sum_{i=1}^{n+1}\abs{a_i}^2\sum_{i=1}^{n+1}\abs{b_i}^2
\end{align*}
so it also holds for \(n+1\). Thus, Cauchy's inequality holds for all \(n\in \N\).
\subsubsection{}
\begin{align*}
	\abs{\sum_{i=1}^n\lambda_ia_i}&\leq\sum_{i=1}^n\abs{\lambda_ia_i}\\
				      &=\sum_{i=1}^n\lambda_{i}\abs{a_i}\\
				      &\leq\sum_{i=1}^n\lambda_i\\
				      &=1
\end{align*}
\subsubsection{}
By the the parallelogram rule and triangle inequalities,
\[4\abs{c}^2 = \abs{z-a}^2+\abs{z+a}^2+2\abs{z^2-a^2} = 2\abs{z}^2+2\abs{a}^2+2\abs{z^2-a^2}\geq  4\abs{a}^2\]
or
\[\abs{c}\geq \abs{a}\]
Thus, there are only solutions for \(\abs{a}\leq \abs{c}\). There are two solutions given by the pair of equations 
\[\abs{z-a}=\abs{c}\qquad\qquad \abs{z+a}=\abs{c}\]
which yield
\[z = \pm i \frac{a}{\abs{a}}\sqrt{\abs{{c}}^2-\abs{a}^2}\]
thus, for \(\abs{a}\leq\abs{c}\) solutions exist. \(\abs{z}\) is bounded above by \(\abs{c}\) like \(\abs{a}\), but can go to zero.

The actual bounds seem too hard to think about right now.
\section{The Geometric Representation of Complex Numbers}
\subsection{Geometric Addition and Multiplication}
\subsubsection{}
\subsubsection{}
\subsubsection{}
\subsubsection{}
\subsection{The Binomial Equation}
\subsubsection{}
From de Moivre's,
\[\cos3\phi = \cos^3\phi - 3\cos\phi\sin^2\phi\]
\[\cos4\phi = \cos^4\phi - 6\cos^2\phi\sin^2\phi+\sin^4\phi\]
\[\sin5\phi = \sin^5\phi-10\sin^3\phi\cos^2\phi+5\sin\phi\cos^4\phi\]
\subsubsection{}
Note we can add the two terms, 
\[\Sigma=1+\cos\varphi+i\sin\varphi+\cdots = 1+e^{i\varphi}+e^{2i\varphi}+\cdots+e^{ni\varphi}\]
yielding
\[\Sigma = \frac{1-e^{in\varphi}}{1-e^{i\phi}} = e^{i\frac{n-1}{2}\phi}\frac{e^{i\frac{n}{2}\phi}-e^{-i\frac{n}{2}\phi}}{e^{-i\phi/2}-e^{-i\phi/2}} = e^{i\frac{n-1}{2}\phi}\frac{\sin\frac{n\phi}{2}}{\sin\frac{\phi}{2}}\]
The cosine terms are the real part of this,
\[\sum_{m=1}^n\cos{m\phi} = \Re\Sigma = \cos\frac{n-1}{2}\phi\frac{\sin\frac{n\phi}{2}}{\sin\frac{\phi}{2}}\]
and the sine terms are the imaginary part of this,
\[\sum_{m=1}^n\sin{m\phi} = \Im\Sigma = \sin\frac{n-1}{2}\phi\frac{\sin\frac{n\phi}{2}}{\sin\frac{\phi}{2}}\]
\subsubsection{}
\subsubsection{}
\[1+\omega^h+\cdots+\omega^{(n-1)h} = \frac{1-\omega^{nh}}{1-\omega^h}=\frac{1-1^h}{1-\omega^h}=0\]
if \(h\) is not divisible by \(n\).
\subsubsection{}
\[1-\omega^h+\cdots+(-1)^{n-1}\omega^{(n-1)h} = \frac{1-(-1)^n\omega^{nh}}{1+\omega^h}=\frac{1-(-1)^n1^h}{1+\omega^h}=\frac{1-(-1)^{n}}{1+\omega^h}\]
\subsection{Analytic Geometry}
\subsubsection{}
When \(a=\pm b\). See 1.1.4.4.
\subsubsection{}
\subsubsection{}
\subsubsection{}
Fix the circle to be at the origin, and set the chords to be between pairs \(z,\bar z\). The midpoints lie on the line \(\Im(z) =0\), which is a diameter, and perpendicular to the chords.
\subsubsection{}
\subsection{The Spherical Representation}
\subsubsection{}
Let
\[z = \frac{x_1+ix_2}{1-x_3}\qquad z' = \frac{-x_1-ix_2}{1-x_3}\]
then,
\[z\bar z' = \frac{-x_1^2-x_2^2}{1-x_3^2}=\frac{-x_1^2-x_2^2}{x_1^2+x_2^2} = -1\]
\subsubsection{}
\subsubsection{}
\subsubsection{}
\subsubsection{}
